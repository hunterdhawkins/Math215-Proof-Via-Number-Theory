\documentclass{article}

\usepackage[english]{babel}
\usepackage[utf8]{inputenc}
\usepackage{amsmath,amssymb}
\usepackage{parskip}
\usepackage{graphicx}
\usepackage{changepage}   % for the adjustwidth environment
\usepackage{verbatim}

% Margins
\usepackage[top=2.5cm, left=3cm, right=3cm, bottom=4.0cm]{geometry}
% Colour table cells
\usepackage[table]{xcolor}

% Get larger line spacing in table
\newcommand\tab[1][1cm]{\hspace*{#1}}

%%%%%%%%%%%%%%%%%
%     Title     %
%%%%%%%%%%%%%%%%%
\title{Math 215: Theorem Packet}
\author{Hunter Hawkins-Stark \\ Vandal Number V00655052}
\date{\today}

\begin{document}
\maketitle

%%%%%%%%%%%%%%%%%%%%%%%%%%%%%%%%%%
%   Proof of Division Algorithm  %
%%%%%%%%%%%%%%%%%%%%%%%%%%%%%%%%%%
\section{Proof of The Division Algorithm}
Claim: Let a,b be integers with b not equal to zero. Then, there exists a unique pair of integers q and r, such that a = b $\cdot$ q +r and 0 $\leq$ r $<$ $\mid$ b $\mid$ \\\\
Proof: Let a and b be integers with b not equal to zero. If b = 1, then q = a, so we will assume that b $>$ 1. Then there exists a set S of the form a-k$\cdot$ b, where k is an integer, and the set S contains all natural numbers that fit the form.

We must now show that S is non-empty, because if S is non-empty the well ordering principle will give us the least element of S. This least element will be r.  


\tab Case 1: a $\geq$ 0, so we will set k =0, plugging into our formula we get $a-0 \cdot b$. The solution is \tab then simply just a, which means that $a \geq 0 $ of S, thus the set S is non-empty.

\tab Case 2: a $<$ 0, so we will set k = a. Plugging into our formula we get $ a - kb$, with k being equal \tab to a we once again substitute to get $a-ab$. Factoring out an a gives us the form $a(1-b)$, and \tab due to a $<$ 0 and b $>$ 1, $a(1-b)$ must be greater than $0$ of S. Thus the set S is non-empty.


With both cases of a either being greater than 0, and being less then zero resulting in a non-empty empty set S, S must have a least element $r$ which is equal to $a-qb$ for some integer q. Thus $a=q \cdot b+r$ and $r \geq 0 $. Now we need to show that r $<$ b, and that q and r are unique. 


Show r $<$ b: Suppose r $\geq$ b, then $r = b+z$, where z is an integer that fits the form of $0 \leq z < r$. Using our original equation $a = q \cdot b + r$ and the fact that $r = b+z$ we perform a substitution for r in the original equation resulting in $a = q \cdot b + b + z$. Simplifying we arrive at the result  $z  = a - (q+1)\cdot b$ which is also an element of our set S, and is smaller than r. Using this we arrive at a contradiction where r is not the least element of S, thus r $<$ b.


Show q and r are unique: Let there exist integers x and y that satisfy $a < xb+ y$ and $0 \leq y < b$. Using the assumption of $y \geq r$ we get $0 \leq r -y < b$. With $xb+y$ being equal to $qb+r$ we get $r-y = b(x-q)$. With us knowing $0 \leq r-y < b$, we then know b divides $r-y$. This $y=r$ and $x = q$ and thus they are unique. 

QED
%%%%%%%%%%%%%%%%%%%%%%%%%%%%%%%%%%%%%%%%%%%%%%%%%%%%%%%%
% Proof of Euclidean Algorithm (updated)
%%%%%%%%%%%%%%%%%%%%%%%%%%%%%%%%%%%%%%%%%%%%%%%%%%%%%%%%
\pagebreak
\section{Proof of the Euclidean Algorithm}
Claim: Suppose a,b $\in \mathbb{N}$. If we repeatedly perform the division algorithm:
\begin{adjustwidth}{1cm}{}
 $a = bq_1 + r_1$\newline
 $b=r_1q_2+r_2$\newline
 $r_1 = r_2q_3+r_3$\newline
 ...\newline
 ...\newline
 $r_{n-3} = r_{n-2} q_{n-1} + r_{n-1}$\newline
 $r_{n-2} = r_{n-1} q_n + 0$\newline
 $r_{n-1} = gcd(a,b)$
 \end{adjustwidth}
Prove that $r_{n-1} = gcd(a,b)$
 
Proof: Let a,b $\in \mathbb{N}$ and d $\in \mathbb{Z}$  First we notice that the sequence $\{r_1,r_2,...\} \subseteq \mathbb{N}$ and is decreasing. Thus it truncates at a certain point, and this point is at $r_{n-1}$. Next we want to show that $r_{n-1} \mid a$ and $r_{n-1} \mid b$. Using the equation $r_{n-2} = r_{n-1} q_n + 0$ we can see $r_{n-1} \mid r_{n-2}$ we can also see that $r_{n-1} \mid r_{n-3}$ which if we continue on we find that $r_{n-1} \mid a$ and $r_{n-1} \mid b$. Now that we know the sequence ends at a point (ie $r_{n-1}$ exists) and that point is a divisor of a and b we need to prove its the greatest common divisor of a and b. Let $d \mid a$ and $d \mid b$. This means that $d \mid a-bq_1$ which when rearranging the equation $a = bq_1 + r_1$ means that $d \mid r_1$ (d divides the first non zero remainder). Also  $d \mid b-r_1q_1$ which once again after rearranging means $d \mid r_2$ (d divides the second non zero remainder). Following this trend we arrive at $d \mid r_{n-1}$ thus, $r_{n-1} = gcd(a,b)$.\newline\newline
QED


\begin{comment}
%%%%%%%%%%%%%%%%%%%%%%%%%%%%%%%%%%%%%%%%%%%%%
%   Proof of Extend Euclidean Algorithm
%   This version is commented out as its
%	inaccurate
%%%%%%%%%%%%%%%%%%%%%%%%%%%%%%%%%%%%%%%%%%%%%
\pagebreak
\section{Proof of the Extended Euclidean Algorithm}
Claim: If d divides a, d dives b, and $d = ax+by$ for some integers x and y, the d = gcd(a,b).

Proof: Let there exist integers a, b,x,y, and d, such that d divides a, d divides b, and d = ax+by. We want to show that d = gcd(a,b) With d dividing both a and b, d cannot exceed the greatest common divisor which means that $d \leq gcd(a,b)$. Also, since the gcd(a,b) is a common factor of a and b, it must also divide $ax+by$. Using this we get $gcd(a,b) \leq d$. After finding the facts that $d \leq gcd(a,b)$ and $gcd(a,b) \leq d$ we can conclude that $d = gcd(a,b)$.

QED

\end{comment}
%%%%%%%%%%%%%%%%%%%%%%%%%%%%%%%%%%%%%%%%%
%   Proof of Fundamental Theorem of Arthmetic 
%%%%%%%%%%%%%%%%%%%%%%%%%%%%%%%%%%%%%%%%%%%
\pagebreak
\section{Proof of the Fundamental Theorem of Arithmetic}
Claim: Let n $\in \mathbb{Z}$ excluding $\{0,+1,-1\}$. Then $\exists$ prime numbers $p_1, ..., p_k$ such that $ n = \prod_{i=1}^{k} p_i$


Proof: This proof will be done by induction, and will use Lemma 1 that states if an integer is greater than 1, then it can be written as a product of primes uniquely apart from the ordering of the primes. Let n $\in \mathbb{Z}$ excluding $\{0,+1,-1\}$
\begin{adjustwidth}{1cm}{}
Base case: Let $n = 2$. Then n can be written as a product of a single prime number which is 2. 

Inductive Step: First we will assume that every integer between 2 and K can be written as a product of one or more primes. Now for induction we need to show that k+1 can be written as a product of primes. 

Case 1: $K+1$ is prime. Then it is the product of one prime, which is itself.

Case 2: $K+1$ is composite. Then $\exists$ a,b which are $\in$ $\{ a,b \in \mathbb{Z} \mid a \geq 2, b \geq 2\}$. By induction a can be written as a product of primes $p_1,..., p_r$ and b can be written as a product of primes $q_1, ..., q_s$ such that $p_1 \leq , ..., \leq p_r$ and $q_1 \leq , ..., \leq q_s$ . With $k+1$ being composite we know it can be written as a product of primes according to lemma 1. In our specific case $k+1$ can be written as $a*b$, which is equivalent to $p_1,...,p_r * q_1,..., q_s$ which is a product of primes. Thus $k+1$ can be represented by a product of primes when composite. 
\end{adjustwidth}
Now that we have shown there exists a solution, we want to show there is a unique prime factorization for n. We will also do this by induction. 
\begin{adjustwidth}{1cm}{}
Let P(k) be the statement that if $p_1,..., p_r = q_1, ..., q_s$ where $p_1 \leq , ..., \leq p_r$ and $q_1 \leq , ..., \leq q_s$ then $r=s$ and $p_i = q_i$ for all i $\in \mathbb{N}$ with $1 \leq i \leq k$

Base case: Show P(1) is a true statement. First let $p_1 = q_1,...,q_s$ where $p_1$ is prime along with $q_1,...,q_s$. With $p_1$ being prime, we have that $s=1$ and thus $p_1 = q_1$.  

Inductive Step: Now we will assume P(k) is true in order to show P(k+1) is true. For P(k+1) we will assume that $p_1,..., p_{k+1} = q_1, ..., q_t$ for the prime numbers $p_1,..., p_{k+1}$ and $q_1, ..., q_t$ that follow the conditions of $p_1 \leq , ..., \leq p_r$ and $q_1 \leq , ..., \leq q_t$. Due to $K \geq 1$ we know $p_1,..., p_{k+1}$ is not prime, and thus $t \geq 2$. Let p be the largest prime such that $p \mid p_1,..., p_{k+1}$, which then follows that $p \mid p_i$ where i is an integer and $1 \leq i \leq K+1$. Since $p_i$ is prime, we know $p = P_i$  which then results in $p = p_i \leq p_{k+1}$. Also in the choice of p, we have $p \geq p_{k+1}$ which follows that $p=p_{k+1}$. By the same logic we know that $ p = q_t$ and thus $ p_{k+1} = q_t$. Then $p_1,..., p_k = q_1, ..., q_{t-1}$. Via the induction hypothesis, we have that $k = t-1$ and that $p_i = q_i$ for all i within $1 \leq i \leq k+1$.

\end{adjustwidth}

Thus we have shown that if P(k) has a solution and that it is unique, then P(k+1) also has a unique solution. By the principle of mathematical induction, P(K) for all k in the natural numbers must be true.

QED

%%%%%%%%%%%%%%%%%%%%%%%%%%%%%%%%%%%%%%%%%%%%%%%
%Euclids proof of Infinitely Many Primes
%%%%%%%%%%%%%%%%%%%%%%%%%%%%%%%%%%%%%%%%%%%%%%%%
\pagebreak
\section{Euclids proof of Infinitely Many Prime Numbers}

Claim: Prove there is infinitely many prime numbers

Proof: This proof will be by contradiction. Let p,k,q $\in \mathbb{Z}$ and suppose for  contradiction that there are only a finite amount of primes. We will write them as $p_1, p_2,..., p_n$. Then let $q = p_1, p_2, ..., p_n +1$. Then q cannot be prime as q $\notin \{p_1,...,p_n\}$, however q will have a prime factor, which we will call $p_k$ where $p_k \in \{p_1,...,p_n\}$. So $p_k$ is a factor of q ,$p_k$ is also a factor of $p_1, p_2,..., p_n$, and thus $p_k$ is a factor of $q-p_1 p_2... p_n$. However since  $\dfrac{q-p_1 p_2... p_n}{p_k} =\dfrac{1}{p_k}$, and $\dfrac{q-p_1 p_2... p_n}{p_k} \in \mathbb{Z}$ but $\dfrac{1}{p_k} \notin \mathbb{Z}$
we have arrived at a contradiction.

Thus there must be infinitely many prime numbers. 

QED

%%%%%%%%%%%%%%%%%%%%%%%%%%%%%%%%%%%%%%%%%%%%%%%%%%
% Show that the operations of addition and mutiplication
% are well defined modulo n
%%%%%%%%%%%%%%%%%%%%%%%%%%%%%%%%%%%%%%%%%%%%%%%%%%%%
\pagebreak
\section{Modular Arithmetic is Well Defined Over Addition and Multiplication}

Addition Claim: If $a \equiv b \; mod \; n$ and  $c \equiv d \; mod \; n$ then $a+c \equiv b+d \; mod \; n$

Addition Proof: Let a,b,c,d,n $\in \mathbb{Z}$. Then by congruence $\exists $ s,t $\in \mathbb{Z}$ such that $a-b = sn$ and $c-d = tn$. Adding these two statements together we get $(a-b) + (c-d) = sn + tn$. Adding $(b+d)$ to each side we get $a+c = b+d + sn + tn$. Factoring n out of the right side we get $a+c = b+d +n(s+t)$. Then by definition of congruence modulo n, $a+c \equiv b+d \; mod \; m$. Thus, addition is well defined modulo n.

QED\newline\newline

Multiplication claim: If $a \equiv b \; mod \; n$ and  $c \equiv d \; mod \; n$ then $a*c \equiv b*d \; mod \; n$

Multiplication Proof: Let a,b,c,d,n $\in \mathbb{Z}$. Then by definition of congruence we know $a = nz +b $ and that $c = ny+d$ for z,y $\in \mathbb{Z}$. Multiplying these two statements together we get $ac = nzny + nzd + nyb + bd$. Then after simplifying and rearranging we get $ac-bd = n(zny + zd + yb)$ which shows $n \mid ac-bd$ and thus by definition of congruence modulo n, $ac \equiv bd \; mod \; n$. This proves that multiplication is well defined modulo m.

QED
%%%%%%%%%%%%%%%%%%%%%%%%%%%%%%%%%%%%%%%%%%%%%%%
%Proof of the Chinese remainder Theorem
%%%%%%%%%%%%%%%%%%%%%%%%%%%%%%%%%%%%%%%%%%%%%%%%
\pagebreak
\section{Proof of the Chinese Remainder Theorem}

Claim: Let r $\in \mathbb{N}$, $n_1, ... , n_r  \in \mathbb{N}$, and $a_1,...,a_r \in \mathbb{Z}$. Let the $gcd(n_i,n_j) =1$ for $i \neq j$. Then the system of congruences

\tab\tab\tab\tab $x \equiv a_i \; mod \; n_i$ where $i \in \{1,...r\}$

has a unique solution module $\prod_{i=1}^{r} n_i $.

Proof: This proof will be done by using induction on r. Let x $\in \mathbb{Z}, r\in \mathbb{N},$ and $a_1,..., a_r \in \mathbb{Z}$. Let Q(r) be the fact that $x \equiv a_i mod n_i$ for $i \in {1,...,r}$ has a solution modulo  $\prod_{i=1}^{r} n_i $ whenever the $gcd(n_i,n_j) =1$ for $i \neq j$.
\begin{adjustwidth}{1cm}{}
Base Case: Let the $gcd(n_i,n_j) = 1$ for all $i \neq j $. With the system of congruences $x \equiv a_i mod n_i$ for $i \in {1,2}$, a solution can only be found if and only if $gcd(n_1,n_2) \mid (a_1-a_2)$. Also that solution modulo $n_1n_2$ is unique only when $gcd(n_1,n_2) = 1$. Thus when $r=2$ the statement holds, which in return proves our base case.

Induction Step: Suppose that $x \equiv a_i mod n_i$ where $i \in$ \{1,...r\} has a solution module $\prod_{i=1}^{r} n_i $.Now using PMI, we must consider the system of congruences with  $x \equiv a_i mod n_i$  where $i \in \{1,...r+1\}$ By the inductive hypothesis there is a solution Y in which $Y \equiv a_i \; mod \; n_i$ for all $i \in \{1,...r+1\}$. Using this fact and the fact that  $x \equiv a_{i+1} \; mod \; n_{r+1}$ we know the two systems of congruences have a unique solution Z. Which means $Z \equiv a_{i} \; mod \; n_{i}$ for all $i \in \{1,...r+1\}$ and Z is a unique solution determined using modulo $\prod_{i=1}^{r+1} n_i $.
\end{adjustwidth}
Thus by the principle of mathematical induction Z has a unique solution. Every other solution to the system is congruent to Z modulo $\prod_{i=1}^{r} n_i $.


QED

%%%%%%%%%%%%%%%%%%%%%%%%%%%%%%%%%%%%%%%%%%%%%%%%%%%%%
% Proof of Eulers Theorem (Revamped)
%%%%%%%%%%%%%%%%%%%%%%%%%%%%%%%%%%%%%%%%%%%%%%%%%%%%%%%
\pagebreak
\section{Proof of Euler's Theorem}
Claim: Let m and a $\in \mathbb{Z}$ and suppose that $m \geq 1$ and $gcd(a,m) = 1$. Then $a^{\phi(m)} \equiv 1 \; mod\;  m$.

Proof: From proving lemma 2 we know that $r_1, r_2, ..., r_{\phi(m)} \equiv (ar_1)(ar_2),...,(ar_{\phi(m)})$ which also means $r_1, r_2, ..., r_{\phi(m)}$ is equivalent to $a^{\phi(m)} (r_1 r_2... r_{\phi(m)}) \; mod \; m$. Due to each of $r_1, r_2, ..., r_{\phi(m)}$ being relatively prime to m, it follows that their product is as well. Hence that factor can be canceled in the last congruence and we get $1 \equiv a^{\phi(m)} \; mod \; m$

QED\newline\newline

\begin{comment}
%%%%%%%%%%%%%%%%%%%%%%%%%%%%%%%%%%%%%%%%%%%%%%%%%%%%%%
% Proof of Eulers Theorem (This is commented out as it
% is an old attempt)
%%%%%%%%%%%%%%%%%%%%%%%%%%%%%%%%%%%%%%%%%%%%%%%%%%%%%%%
\pagebreak
\section{Proof of Euler's Theorem}

Claim: Let n $\geq$ be an integer and a $\in \mathbb{Z}$ with $gcd(a,n) =1 $. Then we wish to prove three core points to get a conclusive proof. First we wish to prove $\{ b\; mod \; n \mid b \in \mathbb{Z}, gcd(b,n)=1\} = \{ab \; mod \; n \mid b \in \mathbb{Z}, gcd(b,n)=1\}$. Secondly is to prove if B = $\{b \mid 1 \leq b \leq n-1 \; with \; gcd(b,n)=1\}$ then \\
$$\prod_{ b \in B} ab  \equiv \prod_{ b \in B} b \; mod \; n $$ After proving these two we will be able to show and prove that $a^{\phi(n)} \equiv 1 \; mod \; n$

Proof: 
For the first section of this proof we want to show equivalence between $\{ b\; mod \; n \mid b \in \mathbb{Z}, gcd(b,n)=1\}$ and $\{ab \; mod \; n \mid b \in \mathbb{Z}, gcd(b,n)=1\}$. Let X= $\{ b\; mod \; n \mid b \in \mathbb{Z}, gcd(b,n)=1\}$ and Y = $\{ab \; mod \; n \mid b \in \mathbb{Z}, gcd(b,n)=1\}$. Then we will number of elements of the sets with the $\phi$ function, which then has the elements of X as $b_1,...,b_{\phi(n)}$ and the elements of Y being $ab_1, ..., ab_{\phi(n)}$. We must now show that the $gcd(ab_i,n) = gcd(b_i, n) = 1$ and that $ab_i \not\equiv ab_j \; mod \; n$ for all $i \neq j \in \{1,...,\phi(n)\}$. For the second condition we will suppose that $ab_i \equiv ab_j \; mod \;n$ then, a is invertible modulo n, $b_i \equiv b_j \; mod \; n$ and that $b_i = b_j$. Further $ab_i \not\equiv ab_j \; mod \; n$ whenever $b_i \neq b_j$. In order to show the $gcd(ab_i,n) = gcd(b_i, n) = 1$, we will assume that $p \mid a$ or $p \mid b_i$ which results in a contradiction as $gcd(ab_i,n) = gcd(b_i, n) = 1$ means we have p = 1, and this is a contradiction. In conclusion we know that all elements of Y are distinct modulo n and that the $gcd(ab_i,n) =1$ for all i, which means X = Y. This proves ( $\{ b\; mod \; n \mid b \in \mathbb{Z}, gcd(b,n)=1\} = \{ab \; mod \; n \mid b \in \mathbb{Z}, gcd(b,n)=1\}$).
\\
\\
Using what we just proved $ab_1,...,ab_{\phi(n)} \equiv  b_1, ..., b_{\phi(n)} mod \; n$, and combining terms we can get $a^{\phi(n)} b_1, ..., b_{\phi(n)} \equiv b_1,...,b_{\phi(n)} mod \; n$. With all $b \in B$ being invertible mod n we can conclude that $a^{\phi(n)} \equiv 1 \; mod \; n$, which thus proves Eulers theorem. 

QED

\end {comment}

%%%%%%%%%%%%%%%%%%%%%%%%%%%%%%%%%%%%%%%%%%%%%%%%%%%%%%
% Proof of Fermats Little Theorem
%%%%%%%%%%%%%%%%%%%%%%%%%%%%%%%%%%%%%%%%%%%%%%%%%%%%%%%
\section{Proof of Fermats Little Theorem (A Special Case of Eulers Theorem)}

Claim: Let n be a prime and $a \in \mathbb{Z}$. Assume $n \nmid a$, then $a^{n-1} \equiv 1  \; (mod \; n)$

Proof: Using Eulers theorem (proven above) we have the formula $a^{\phi(n)} \equiv 1 \; mod \; n$. In our case we are dealing with a prime number, which gives us $\phi(n) = n-1$. Substituting this into our equation we get $a^{n-1} \equiv 1 \; mod \; n$. Which then proves Fermats little theorem. 

QED

%%%%%%%%%%%%%%%%%%%%%%%%%%%%%%%%%%%%%%%%%%%%%%%%%%%%%%%%%%%
%There are phi(p-1) primitive roots modulo a prime number p
%%%%%%%%%%%%%%%%%%%%%%%%%%%%%%%%%%%%%%%%%%%%%%%%%%%%%%%%%%%%%%%
\pagebreak
\section{There are $\phi(p-1)$ primitive roots mod p}

Claim:Every prime p has $\phi(p-1)$ primitive roots.

Proof: Lemma 6 states that each integer $1,2,...,p-1$ has an order that is a divisor of $p-1$. For each divisor r of $p-1$, let $\psi(r)$ represent the number of integers in $1,2,...,p-1$ that have order r. From Lemma 7 we know that $\sum\limits_{r \mid p-1} \psi(r) = \sum\limits_{r \mid p-1} \phi(r)$. It will follow from this equation that if we can show $\psi(r) \leq \phi(r)$ we can conclude that $\psi(r) = \phi(r)$ for each r. The number of primitive roots of p will be $\psi(p-1) = \phi(p-1)$. From here we will choose an r.
\begin{adjustwidth}{1cm}{}
Case 1: if $\psi(r) =0$ then $\psi(r) < \phi(r)$ and we have proved what we needed to prove. 

Case 2: If  $\psi(r) \neq 0$, then there is an integer with order r, which we will call a. The congruence $x^r \equiv 1 \; mod \; p$ has exactly r solutions according to lemma 4. Also, $x^r \equiv 1 \; mod \; p$ is satisfied by the r integers $ a, a^2, a^3, ..., a^r$, where they all give solutions as no two of these have the same least residue mod p. From lemma 5, the numbers in $a, a^2, a^3, ..., a^r$ have order r are those powers $a^k$ with the $gcd(k,r) =1$. However, there are $\phi(r)$ such numbers k. Hence $\psi(r) = \phi(r)$ in this case. 
\end{adjustwidth}

Thus, we know every prime number p has $\phi(p-1)$ primitive roots.

QED

%%%%%%%%%%%%%%%%%%%%%%%%%%%%%%%%%%%%%%%%%%%%%%%%%%%%%%%%%%%%%%%%
% There are either 2 solutions or no solutions to the congruence
% x^2 \equiv a \bmod p for a prime number p and an integer a with 
% p dividing a. Further the congruence has solutions iff
% a^((p-1)/2) \equiv 1 mod p.
%%%%%%%%%%%%%%%%%%%%%%%%%%%%%%%%%%%%%%%%%%%%%%%%%%%%%%%%%%%%%
\pagebreak
\section{There are 2 solutions or no solutions to the congruence $x^2 \equiv a \bmod p$ with $p \mid a$}

Claim: There are either 2 solutions or no solutions to the congruence $x^2 \equiv a \bmod p$ for a prime number p and an integer a with p $\mid$ a. Further the congruence has solutions iff $ a^{(p-1)/2}\equiv 1 \bmod p$

proof: Let p be an odd prime number, let g be a primitive root, and let x,l $\in \mathbb{Z}$. First we should move a to the other side of the congruence which gives us $x^2-a \equiv 0 \; mod \; p$. From here we know by definition of congruence that $p \mid x^2 -a$, which is the same as $p \mid (x -a)*(x+a)$. Then due to p being prime we know that  $p \mid (x -a)$ or  $p \mid (x+a)$. Thus by definition of congruence x must have two or no solutions as $x \equiv a \; mod \; p$ or $x \equiv -a \; mod \; p$. a is a square modulus p if and only $a \equiv g^{2l}$. Thus $a^{(p-1) \mid 2} \equiv (g^{2l})^{(p-1) \mid 2} \equiv (g^{p-1})^l \equiv 1$ If a is a square and $a^{(p-1) \mid 2} \equiv (g^{2l+1})^{(p-1)\mid 2} \equiv (g^{p-1})^l * g^{(p-1)/2} \equiv -1$

Thus there are either 2 solutions or no solutions to the congruence $x^2 \equiv a \bmod p$ for a prime number p and an integer a with p $\mid$ a, and further the congruence has solutions iff $ a^{(p-1)/2}\equiv 1 \bmod p$.

QED


%%%%%%%%%%%%%%%%%%%%%%%%%%%%%%%%%%%%%%%%%%%%%
% Lemma used to assist in the F.T.A
%%%%%%%%%%%%%%%%%%%%%%%%%%%%%%%%%%%%%%%%%%%% 
\pagebreak
\section{Lemmas}
Lemma 1: Let n be an integer. Every $n >1$ is equal to a product of prime numbers (potentially just one prime number).
\begin{adjustwidth}{1cm}{}
Proof: This proof is by contradiction. Assume that there is at least one integer greater than 1, we will call it M, that is not equal to the product of primes. Since M is not equal to the product of prime numbers, then M must be composite. Then $\exists$ a,b $\in \mathbb{Z}$ such that $m=ab$, $1<a<m$, and $1<b<m$. Although, m was the smallest integer greater than one that was not equivalent to a product of primes, and thus a and b must be equivalent to a product of primes. Hence $m=a*b$ has to be equivalent to a product of primes, and thus leads us to a contradiction. 
\end{adjustwidth}
Thus we know ever integer that is greater than 1 is equivalent to the product of prime numbers (it may potentially be one).

QED\newline\newline

%%%%%%%%%%%%%%%%%%%%%%%%%%%%%%%%%%%%%%%%%%%%%%%%%%%%
% Lemma used to help proof of Eulers Theorem
%%%%%%%%%%%%%%%%%%%%%%%%%%%%%%%%%%%%%%%%%%%%%%%%%%
Lemma 2: If the $gcd(a,m)=1$ and $r_1, r_2, ..., r_{\phi(m)} $ are the positive integers less than m and relatively prime to m, then the least residues mod m of $ar_1, ar_2, ..., ar_{\phi(m)}$ are a permutation of $r_1, r_2,..., r_{\phi(m)}$.

\begin{adjustwidth}{1cm}{}
Proof: Do to their being exactly $\phi(m)$ numbers in the set, to prove that their least residues are a permutation of $\phi(m)$ numbers $r_1, r_2,..., r_{\phi(m)}$. We have to show that they are all relatively prime to m and show that they are all different. 

Prove all numbers are relatively prime to m: Let p be a prime number that is a common divisor of $ar_i$ and m for some i where $1 \leq i \leq \phi(m)$. With p being prime, we know either p $\mid$ a or p $\mid $ $r_i$. This tells us that either p is a common divisor of a and m or p is a common divisor of m and $r_i$. However, with the $gcd(r_i, m) =1$ and the $gcd(a,m) =1$ this means the possibility of p being a common divisor in any case is a contradiction. Thus the $gcd(ar_i, m) = 1$ for each i $1\leq i \leq \phi(m)$

Prove all are different: Let i and j $\in \mathbb{Z}$ where $1 \leq i \leq \phi(m)$ and $1 \leq j \leq \phi(m)$ such that $ar_i \equiv ar_j \; mod \; m$. Due to the $gcd(a,m)$ being one, we can cancel a from both sides in order to get $r_i \equiv r_j \; mod \; m$. With $r_i = r_j$ we know that if $r_i \neq r_j$ then $ar_i \not\equiv ar_j \; mod \; m$ and thus all the numbers are different. 
\end{adjustwidth}

Thus we know that the least residues mod m of $ar_1, ar_2, ..., ar_{\phi(m)}$ are a permutation of $r_1, r_2,..., r_{\phi(m)}$.

QED\newline\newline
\newpage
Lemma 3: If f is a polynomial of degree n, then $f(x) \equiv 0 \; mod \; p$ has at most n solutions
\begin{adjustwidth}{1cm}{}
Proof:This proof will be on induction on the degree n. Let $f(x) = a_{n}x^n + a_{n-1}x^{n-1}+...+ a_0$ have degree n so that $a_n \not\equiv 0 \; mod \;p$. 

Base Case: For $n=1$ $a_1x + a_0 \equiv 0 \; mod \; p$. has only one solution since the $gcd(a1,p) =1$.

Inductive Hypothesis: Let the lemma be true for polynomials of degree n-1, and let f have degree n. Then Either $f(x) \equiv 0 \; mod \; p$ has no solutions or it has atleast one solution.

Case 1:  $f(x) \equiv 0 \; mod \; p$ has no solutions, thus the lemma is true.

Case 2: Let y be a solution such that $f(y) \equiv 0 \; mod \; p$ where y is a least residue mod p. Then due to $x-y$ being a factor of $x^z - y^z$ for $z = 0,1,...n$ we have $f(x) \equiv f(x)-f(y)$. After substituting and simplifying we end up with $f(x) \equiv a_n(x^n-y^n) + a_{n-1}(x^{n-1}-y^{n-1}) +...+a_1(x-y)$. Finally we get $f(x) \equiv (x-y) h(x) \; mod \; p$ where h is a polynomial of degree n-1. If we let w be a solution of $f(x) \equiv 0 \; mod \; p$ we get $f(w) \equiv (w-y)h(w) \equiv 0 \; mod \;p$. Due to p being prime we know $w \equiv y \; mod \; p$ or $h(w) = 0 mod p$. From the induction assumption, the second congruence has at most n-1 solutions. Due to the first congruence having just one solution, we have proved $f(x) \equiv 0 \; mod \; p$ has at most n solutions.
\end{adjustwidth}
Thus by PMI, If f is a polynomial of degree n, then $f(x) \equiv 0 \; mod \; p$ has at most n solutions.

QED\newline\newline

Lemma 4: If $d \mid p-1$, then $x^d \equiv 1 \; mod \; p$ has exactly d solutions. 
\begin{adjustwidth}{1cm}{}
Proof: Using Fermats theorem, $x^{p-1} \equiv 1 \; mod \; p$ has exactly $p-1$ solutions, specifically $1,2,...,p-1$. Further, $x^{p-1} -1 = (x^d -1)(x^{p-1-d}+x^{p-1-2d}+...+1)$, then $x^{p-1}-1 = (x^d-1)g(x)$. From our third lemma we know that $g(x) \equiv 0 \; mod \;p$ has at most $p-1-d$ solutions. Hence $x^d \equiv 1 \; mod \;p$ has atleast d solutions. If we apply lemma 3 again we see that $x^d \equiv 1 \; mod \;p$ has exactly d solutions. 
\end{adjustwidth}
Thus, If $d \mid p-1$, then $x^d \equiv 1 \; mod \; p$ has exactly d solutions. 


QED\newline\newline

Lemma 5: Suppose that a has order t mod m. Then $a^k$ has order t mod m if and only if the $gcd(k,t) =1$.
\begin{adjustwidth}{1cm}{}
Proof: Suppose that the $gcd(k,t) =1$ and denote the order of $a^k$ by s. We have that $1 \equiv (a^t)^k \equiv (a^k)^t \; mod \; m$, so we know $s \mid t$ as s is the order of $a^k$. Then we have $(a^k)^s \equiv a^{ks} \equiv 1 \; mod \; m.$ We know $t \mid ks$ due to the $gcd(k,t)=1$ it follows that $t \mid s$. This fact, along with $s \mid t$ implies that $s=t$. To prove the converse, let a and $a^k$ have order t and that the $gcd(k,t) =r$. Then $1 \equiv a^t \equiv (a^t)^{t \mid r} \; mod \; m$. Due to the order of $a^k$ we know that $t \mid r$ is a multiple of t which implies that $r=1$.
\end{adjustwidth}

Thus, If a has order t mod m, then $a^k$ has order t mod m if and only if the $gcd(k,t) =1$.

QED\newline\newline
\newpage

Lemma 6: If $gcd(a,m) =1$ and a has order t mod m, then $t|\phi(m)$.
\begin{adjustwidth}{1cm}{}
Proof: Using Eulers extension of Fermats Theorem we know that $a^{\phi(m)} \equiv 1 \; mod \; m$. Which allows us to conclude $\phi(m)$ is a multiple of t. Thus we know that $t \mid \phi(m)$
\end{adjustwidth}
Thus, If $gcd(a,m) =1$ and a has order t mod m, then $t|\phi(m)$.

QED\newline\newline

Lemma 7: If $n \geq 1$ then, $\sum\limits_{d \mid n} \phi(d) = n$
\begin{adjustwidth}{1cm}{}
Proof: We have m in $c_d$ if and only if the $gcd(m,n) =d$. However, the $gcd(m,n) = d if and only if (m \mid d, n \mid d) =1$. This means an integer m is in class $C_d$ if and only if $m \mid d$ is relatively prime to $n/d$. The number of positive integers $\leq n \mid d$ and relatively prime to $n \mid d$ is $\phi(n \ mid d)$ by definition. This the number of elements in class $C_d$ is $\phi(n \mid d)$.
\end{adjustwidth}
Thus, If $n \geq 1$ then, $\sum\limits_{d \mid n} \phi(d) = n$

QED\newline\newline



\end{document}
